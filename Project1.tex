\documentclass[11pt,a4paper]{report}

\usepackage[utf8]{inputenc}
\usepackage[T1]{fontenc}
\usepackage[english,norsk]{babel}
\usepackage{amsmath}

\pagestyle{empty}

%% Numbered exercises
\newcounter{excount}[chapter]
\newenvironment{exercise}[1][]{\addtocounter{excount}{1} \noindent {\bf Oppgave
    \arabic{excount} \ \ #1}\hspace{2mm}}{\vspace{4mm}}

\title{FYS2140 Kvantefysikk, Oblig 10}


\author{Mikkel Killingmoe Christensen, gruppe 2}


\begin{document}

\maketitle



\noindent
Dette er nok en oblig som dreier seg om hydrogenatomet og er en del av en tidligere eksamensoppgave.

\bigskip

\begin{exercise}[\\]
I denne oppgaven ser vi bort fra elektronets egenspinn. H-atomet kan da beskrives ved tilstandsfunksjonene $\psi_{n l m_{l}}(\vec{r})$ som har f{\o}lgende egenskaper:
\begin{eqnarray}
\hat{H}_0 \psi_{n l m_{l}}(\vec{r}) &=&
E_{n} \psi_{n l m_{l}}(\vec{r}),                   \\
\hat{L}^2 \psi_{n l m_{l}}(\vec{r}) &=&
\hbar^2 l(l+1) \psi_{n l m_{l}}(\vec{r}),          \\
\hat{L}_{z} \psi_{n l m_{l}}(\vec{r}) &=&
\hbar m_l \psi_{n l m_{l}}(\vec{r}),                \\
\int \psi^{*}_{n l m_{l}}(\vec{r}) \psi_{n' l' m'_{l}}(\vec{r}) \, d^3\vec{r} &=&
\delta_{n,n'} \delta_{l,l'} \delta_{m_l,m_l'},
\end{eqnarray}
hvor
\begin{equation}
\delta_{k,k^{'}} = \left \{ \begin{array}{ll} 1& {\rm for~} k = k'\\
															  0 & {\rm ellers.}\\
										 \end{array}
							\right . \nonumber
\end{equation}
\begin{itemize}
\item[\bf a)] Hva kaller vi disse (to) typene ligninger?\\
\\
\textbf{Svar:}\\
\textbf{De tre første likningene kalles egenverdiligninger, og den siste er et ortonormalitetsintegral. }
\end{itemize}

\noindent
Operatoren $\hat{H}_{0}$ er tidsuavhengig. Kvantetallet $n$ kan anta verdiene $1, 2,\ldots$. For en gitt verdi av $n$ kan $l$ anta verdiene $0, 1, \ldots , n-1$, og $m_{l}$ kan for en gitt verdi av $l$ anta verdiene $-l, -l+1, \ldots , l-1, l$.

I denne oppgaven trengs ingen andre opplysninger enn de som er gitt ovenfor. Det er ikke n{\o}dvendig {\aa} kjenne de eksplisitte uttrykkene for operatorene $\hat{H}_0$, $ \hat{L}^2$ og $\hat{L}_z$. Det skal ikke tas hensyn til elektronets egenspinn.
\begin{itemize}
%
\item[\bf b)] Hvilke fysiske st{\o}rrelser er representert ved operatorene $\hat{H}_{0}$, $\hat{L}^2$ og $\hat{L}_z$?\\
\\
\textbf{Svar:\\
$\hat{H}_{0}$ representerer energien, $\hat{L}^2$ representerer kvadratet av angulærmomentet og $\hat{L}_z$ representerer z-komponenten til angulærmomentet.}

\item[\bf c)] Hvilke fysiske st{\o}rrelser har skarpe verdier i tilstanden $\psi_{n l m_{l}}(\vec{r})$?\\
\\
\textbf{Svar:\\
Det kan vises at operatorene $\hat{H}_{0}$, $\hat{L}^2$ og $\hat{L}_z$ kommuterer. Det medfører at de har de skarpe verdiene E, $L^2$ og $L_z$}

\end{itemize}

\noindent
Ved tiden $t = 0$ er H-atomets tilstand beskrevet ved tilstandsfunksjonen
$\psi_{n l m_{l}}(\vec{r})$. Tidsutviklingen av tilstandsfunksjonen er bestemt
av den tidsavhengige Schr\"{o}dingerligningen
\begin{equation}
\hat{H}_{0} \Psi (\vec{r}, t) = i\hbar \frac{\partial }{\partial t} \Psi (\vec{r}, t).
\end{equation}
\begin{itemize}
\item[\bf d)] Bestem tilstandsfunksjonen som beskriver H-atomets tilstand ved tiden $t$.\\
\\
\textbf{Svar:}\\
Tilstandsfunksjonen er gitt som:
\begin{equation}
\Psi(\vec{r},t) = \sum_{k}c_{k}\psi_{k}(\vec{r})e^{-\frac{i}{\hbar}E_{k}t}
\end{equation}
hvor $c_{k}$ er gitt som:
\begin{equation}
c_{k} = \int\psi_{k}^{\ast}(\vec{r})\psi(\vec{r})d\vec{r}
\end{equation}
Dette gir (ved å utnytte ortonormalitetsintegralet (4)) den endelige tilstandsfunksjonen:
\begin{equation}
\bf \Psi(\vec{r},t) = \psi_{nlm_{l}}(\vec{r})e^{-\frac{i}{\hbar}E_{n}t}
\end{equation}

\end{itemize}

\noindent
\\
La H-atomets tilstand ved tiden $t = 0$ n{\aa} v{\ae}re gitt ved
tilstandsfunksjonen
\begin{eqnarray}
\Phi (\vec{r}) = \frac{1}{\sqrt{2l+1}} \sum_{m_{l} = -l}^{l}
\psi_{n l m_{l}}(\vec{r}).
\label{a}
\end{eqnarray}
\begin{itemize}
\item[\bf e)] Vis at $\Phi (\vec{r})$ er normert.\\
\\
\textbf{Svar:}\\
\begin{equation}
\int\int\int\Phi^{\ast}(\vec{r})\Phi(\vec{r})d^{3}\vec{r}
\end{equation}
\begin{equation}
= \int_{0}^{\infty}\int_{0}^{\pi}\int_{0}^{2\pi}\frac{1}{\sqrt{2l+1}}\sum_{m_{l}=-l}^{l}\psi^{\ast}_{nlm_{l}}(\vec{r})\frac{1}{\sqrt{2l+1}}\psi_{n'l'm_{l}'}(\vec{r})d\phi d\theta dr
\end{equation}
(utnytter ortonormalitetsintegralet (4) og ser at noen integraler = 0)
\begin{equation}
=\frac{1}{2l+1}\sum_{m_{l}=-l}^{l}\int_{0}^{\infty}\int_{0}^{\pi}\int_{0}^{2\pi}\psi^{\ast}_{nlm_{l}}(\vec{r})\psi_{n'l'm_{l}'}(\vec{r})d^{3}\vec{r}
\end{equation}
(integralet i (12) blir antall $m_{l}$ som er lik $2l+1$)
\begin{equation}
=\frac{2l+1}{2l+1} = \bf 1
\end{equation}


\item[\bf f)] Vis at tilstandsfunksjonen ved tiden $t$ er
\begin{eqnarray}
\Psi (\vec{r}, t) = \Phi (\vec{r}) \exp{\left( -\frac{i}{\hbar } E_{n} t \right)}.
\label{b}
\end{eqnarray}\\
\\
\textbf{Svar:}\\
Har fra likning (6) at tilstandsfunksjonen er gitt ved:
\begin{equation}
\Psi(\vec{r},t) = \sum_{k}c_{k}\psi_{k}(\vec{r})e^{-\frac{i}{\hbar}E_{k}t}
\end{equation}
hvor
\begin{equation}
|c_{k}|^{2} = \langle\psi^{\ast}_{k}|\Phi\rangle
\end{equation}
Dette gir at:
\begin{equation}
c_{nlm_{l}}=\frac{1}{\sqrt{2l+1}}
\end{equation}
som insatt i (15) gir:
\begin{equation}
\Psi(\vec{r},t) = \sum_{m_{l}=-l}^{l}\frac{1}{\sqrt{2l+1}}\psi_{nlm_{l}}(\vec{r})e^{-\frac{i}{\hbar}E_{n}t}
\end{equation}
Setter inn for $\Phi$ fra likning (9) og får:
\begin{equation}
\bf\Psi(\vec{r},t) = \Phi(\vec{r})e^{-\frac{i}{\hbar}E_{n}t}
\end{equation}

\item[\bf g)] Bestem forventningsverdien for operatorene $\hat{H}_{0}$, $\hat{L}^2$ og $\hat{L}_z$ i tilstanden som er beskrevet ved tilstandsfunksjonen $\Psi (\vec{r}, t)$ i ligning~(\ref{b}).\\
\\
\textbf{Svar:}\\
Når en går fra $\psi_{nlm_{l}}(\vec{r})$ til $\Phi(\vec{r})$ endres kun $m_{l}$. Dette påvirker kun operatoren $\hat{L}_{z}$.
\begin{equation}
\langle\hat{H_{0}}\rangle = \langle\psi|\hat{H}\psi\rangle = E_{n}
\end{equation}
\begin{equation}
\langle\hat{L}^{2}\rangle = \langle\psi|\hat{L}^{2}\psi\rangle = \bf \hbar^{2}l(l+1)
\end{equation}
\begin{equation}
\langle\hat{L}_{z}\rangle = \langle\psi|\hat{L}_{z}\psi|\rangle = \frac{\hbar}{2l+1}\sum_{m_{l}=-l}^{l}m_{l} = \bf 0
\end{equation}
(fordi $\sum_{m_{l}=-l}^{l}m_{l}$ = 0)

\end{itemize}

\noindent
En st{\o}rrelse $A$ er representert ved operatoren $\hat{A}$. Spredningen $\sigma _A$ i tilstanden $\Psi$ er definert slik:
\begin{equation}
\sigma_ A = \sqrt{\langle \hat{A}^2 \rangle - \langle \hat{A} \rangle^2}.
\end{equation}
\begin{itemize}
\item[\bf h)] Finn spredningen av st{\o}rrelsene representert ved operatorene $\hat{H}_{0}$, $\hat{L}^2$ og $\hat{L}_z$ i tilstanden som er beskrevet ved tilstandsfunksjonen $\Psi (\vec{r}, t)$ i ligning~(\ref{b}).\\
\\
\textbf{Svar:}\\
\begin{equation}
\sigma_{H_{0}}=\sqrt{\langle\hat{H_{0}}^{2}\rangle-\langle\hat{H_{0}}\rangle^{2}}=\sqrt{E_{n}-E_{n}}=\bf 0
\end{equation}
\begin{equation}
\sigma_{L^{2}} = \sqrt{\langle\hat{L}^{2^{2}}\rangle-\langle\hat{L}^{2}\rangle^{2}}=\sqrt{\hbar^{4}l^{2}(l+1)^{2}-\hbar^{4}l^{2}(l+1)^{2}} = \bf 0
\end{equation}
\begin{equation}
\sigma_{L_{z}} = \sqrt{\langle\hat{L^{2}}_{z}\rangle-\langle\hat{L}_{z}\rangle^{2}} = \sqrt{\frac{\hbar^{2}}{2l+1}\sum_{m_{l}=l}^{l}m_{l}^{2}-0}
\end{equation}
\begin{equation}
= \frac{\hbar}{\sqrt{2l+1}}\sqrt{\sum_{m_{l}=l}^{l}m_{l}^{2}}=\frac{\hbar}{\sqrt{2l+1}}\sqrt{2\sum_{m_{l}=0}^{l}m_{l}^{2}}
\end{equation}
\begin{equation}
= \frac{\hbar}{\sqrt{2l+1}}\sqrt{2\frac{l(l+1)(2l+1)}{6}}=\bf \hbar\frac{l(l+1)}{3}
\end{equation}
hvor det har blitt brukt at $\sum_{m_{l}=0}^{l}m_{l}^{2}$ kan skrives som $\frac{l(l+1)(2l+1)}{6}$ fra Rottmann s.111.
\end{itemize}

\noindent
La H-atomets tilstand ved tiden $t = 0$ v{\ae}re gitt ved tilstandsfunksjonen $\Phi (\vec{r})$ i ligning~(\ref{a}) og la oss tenke oss at vi foretar en id\'eell
m\aa ling av $L_{z}$.
\begin{itemize}
\item[\bf i)] Hvor stor er sannsynligheten for {\aa} observere den bestemte verdien
$\hbar  m_{l} $ for $L_z$ ved tiden $t = 0$?\\
\\
\textbf{Svar:}\\
Sannsynligheten er gitt ved formelen $|c_{k}|^{2}$.\\
Dette gir at:
\begin{equation}
|c_{m_{l}}|^{2}=|\frac{1}{\sqrt{2l+1}}|^{2}=\bf \frac{1}{2l+1}
\end{equation}


\item[\bf j)] Vil denne sannsynligheten v{\ae}re avhengig av ved hvilken tid $t > 0$
m\aa lingen utf\o res?\\
\\
\textbf{Svar:\\
Vi kan se at formelen $\Psi(\vec{r},t) = \sum_{k}c_{k}\psi_{k}(\vec{r})e^{-\frac{i}{\hbar}E_{k}t}$ gjelder for alle t. Dette medfører at sannsynligheten ikke vil være avhengig av hvilken tid vi måler den ved.}


%
\end{itemize}
%

\noindent
Vi lar n{\aa} H-atomet befinne seg i et homogent magnetfelt $B$ og velger $z$-aksen langs magnetfeltet. Hamilton operatoren for systemet er da
\begin{eqnarray}
\hat{H} = \hat{H}_{0} + \frac{e}{2m} B \hat{L}_{z},
\end{eqnarray}
der $-e$ er elektronets ladning og $m$ er elektronets masse.
\begin{itemize}
\item[\bf k)] Bestem H-atomets energi i tilstanden $\psi_{n l m_{l}}(\vec{r})$.\\
\\
\textbf{Svar:}\\
\begin{equation}
\hat{H}\psi_{nlm_{l}}(\vec{r})=E_{tot}\psi_{nlm_{l}}
\end{equation}
\begin{equation}
E_{tot}\psi_{nlm_{l}}(\vec{r})=\bigg[\hat{H}_{0}+\frac{e}{2m}B\hat{L}_{z}\bigg]\psi_{nlm_{l}} = E_{n}\psi_{nlm_{l}}+\frac{e}{2m}B\hat{L}_{z}\psi_{nlm_{l}}
\end{equation}
Deler så på $\psi_{nlm_{l}}$ og får:
\begin{equation}
\Rightarrow \bf E_{tot} = E_{n} + \frac{e}{2m}B\hat{L}_{z}
\end{equation}

\item[\bf l)] Er tilstanden $\Phi (\vec{r})$ i ligning~(\ref{a}) en energi-egentilstand for $\hat{H}$? Begrunn svaret.\\
\\
\textbf{Svar:}\\
Fordi $\Phi(\vec{r})$ er en lineærkombinasjon av $\psi_{nlm_{l}}$ som igjen er energi-egentilstander for $\hat{H}$, er også $\Phi(\vec{r})$ dette. 

\item[\bf m)] Bestem forventningsverdien til $\hat{H}$ i tilstanden $\Phi (\vec{r})$.\\
\\
\textbf{Svar:}\\
\begin{equation}
\langle\hat{H}\rangle=\langle\Phi|\hat{H}\Phi\rangle = \langle\Phi|(\hat{H}_{0}+\frac{e}{2m}B\hat{L}_{z})\Phi\rangle = \langle\Phi|\hat{H}_{0}\Phi\rangle+\langle\Phi|\frac{e}{2m}B\hat{L}_{z}\Phi\rangle
\end{equation}
\begin{equation}
= E_{n} + \frac{eB\hbar}{2m(2l+1)}\sum_{m_{l}=-l}^{l}m_{l} = \bf E_{n}
\end{equation}

\end{itemize}
\end{exercise}


\end{document}